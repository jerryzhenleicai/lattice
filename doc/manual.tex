\documentclass[12pt]{article}


\usepackage{verbatim}   % useful for program listings
\usepackage{hyperref}   % use for hypertext links, including those to external documents and URLs

% don't need the following. simply use defaults
\setlength{\baselineskip}{16.0pt}    % 16 pt usual spacing between lines

\setlength{\parskip}{3pt plus 2pt}
\setlength{\parindent}{20pt}
\setlength{\oddsidemargin}{0.5cm}
\setlength{\evensidemargin}{0.5cm}
\setlength{\marginparsep}{0.75cm}
\setlength{\marginparwidth}{2.5cm}
\setlength{\marginparpush}{1.0cm}
\setlength{\textwidth}{150mm}


% above is the preamble

\begin{document}

\begin{center}
{\large Lattice: A Multi-module Java Build System written in Python (Draft)} \\
\copyright 2010 by Zhenlei Cai \\
December 31, 2010
\end{center}

\section{Overview}
In Lattice build files are written not in XML, but in the Python
language. The benefits are much better readability and powerful
imperative build scripting supported by Python. 

For multi-module projects. Lattice uses topological sorting to decide
the correct order to build each module.  It's also planned that
Lattice will analyze the module
dependency to determine how the module compilation can be
parallelized.

Lattice's source code is extremely lean, currently it consists of
about 500 lines of Python source code.

Lattice supports generating Eclipse projects from the build files.


\section{Five Minute  Introduction}
A simple Lattice build file can be as short as two lines:

\begin{verbatim}
# build file module1/__init__.py
depends = [ 'module2']
libs = ['logging', 'commons']
\end{verbatim}

A typical module is a directory that contains Java source files which
are built into a single JAR file. A library is a directory that
contains one or more  third party binary JAR files required for
compiling or running modules. Lattice's build files are named {\it  \_\_init\_\_.py}. 
Here   {\it module1/\_\_init\_\_.py}  states module1 depends on another
module (module2) 
and two libraries: logging and commons.  On disk the files layout for such a two module project is:

\begin{verbatim}
    -- module1 --+-- __init__.py
                 |
                 +- src / main /java /  (Java sources)

    -- module2 --+-- __init__.py
                 |
                 +- src / main /java / ... (Java sources)

    -- libs -----+--- logging / log4j.jar, slf4j-api.jar, slf4j-log4j.jar ...
                 |
                 +-- commons / commons-lang.jar, commons-io.jar, commons-codec.jar ...
\end{verbatim}
Here {\it logging} is a user defined library (UDL) that consists of log4j and SLF4J's binding for log4j. {\it commons} is another  UDL that contains a few Apache Commons libraries. 

To build module1 and module2, type:

{\it   lat -m module1 -t jar\_all} 

This will compile all the Java source code in module2 and then module1
and create \emph{module1/target/module1.jar} and module2/target/module2.jar


\section{Tasks}

A typical Lattice command is in the form of :

\begin{verbatim}
lat -m <module> -t <task>
\end{verbatim}

Lattice has several system built-in tasks: {\it
  build,clean,clean\_local,jar,jar\_all, junit, run\_java\_class,
  war
}.
It's also easy to have per-module user-defined tasks, for example, you
can add a Python function to the build file:

\begin{verbatim}
def run_multiple_java_programs ():
    tasks.run_java_class(__name__, 'com.mycompany.JavaProg1', 'arg1',
mx='1024m')
    tasks.run_java_class(__name__, 'com.mycompany.JavaProg2', 'arg2',
mx='512m')
\end{verbatim}

Here a custom task {\tt run\_multiple\_java\_programs}  is defined to run two Java classes with specific
arguments and JVM memory settings. The classpath settings needed for
running the classes will be automatically calculated by Lattice. The
task can be invoked as:
\begin{verbatim}
lat -m mymodule -t run_multiple_java_programs
\end{verbatim}
Because a custom task is just a regular Python function, and Python is
a very high-level language well suited for invoking operating system
commands, shell scripts and other programs , a
custom task can be created to do almost any types of work. It should
be easy to invoke other Java build systems such as ant, maven or ivy in
a custom task.

\section{Installation}

Source code:  {\tt https://github.com/hackingspirit/lattice}  .
To download a tar ball:
https://github.com/hackingspirit/lattice/tarball/master  .
To check out a read-only copy: 

\begin{verbatim}
git clone git://github.com/hackingspirit/lattice.git
\end{verbatim}


To install:
\begin{verbatim}
sudo python setup.py install
\end{verbatim}
This will install lattice into your Python system path. 
To run Lattice, copy the shell script "lat"  to a directory that's in your executable search PATH, such as /usr/bin .   Type "lat -h" to see a list of options.

\section{A multi-module Java Web Project Example}
A seven-module sample project where the topmost module builds into a .war
file is included in the source distribution. For details 
please see README file under the sample\_java\_prj/ folder of the distribution.

\section{Eclipse Support}
To generate Eclipse projects:
\begin{verbatim}
lat --eclipse-gen 
\end{verbatim}

Each module or library is converted to one Eclipse project.  In
addition a projectList.txt file is created which can be used to import
all the projects into Eclipse at once using the
excellent Eclipse bulk import tool
(http://code.nomad-labs.com/eclipse-bulk-import/) .


\section{Some Build Details}
The default task is {\it build}, when executed Lattice compiles all the dependent
modules transitively first (unless the {\it
  --disable-transitive-dependencies} flag is on) and then compiles the
module being built. A Java source file (.java) is only (re)compiled if
it's newer than its corresponding .class file.  

By default Java sources are found in {\tt src/main/java}, resource
files are under {\tt src/main/resources}, web files (JSP, HTML etc)
are under {\tt src/main/web}, JAR's and WAR's are built into {\tt
  ./target/} and intermediate files (.class etc) are saved in {\tt build/}.  
These settings are defined in the file {\it settings.py} and can be easily customized.



\section{Future Work}

Parallel builds, detect library collisions (different versions of same
library/package etc), various plugin-tasks (GWT, ...)


\end{document}